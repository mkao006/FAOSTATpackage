\documentclass{article}\usepackage{graphicx, color}
%% maxwidth is the original width if it is less than linewidth
%% otherwise use linewidth (to make sure the graphics do not exceed the margin)
\makeatletter
\def\maxwidth{ %
  \ifdim\Gin@nat@width>\linewidth
    \linewidth
  \else
    \Gin@nat@width
  \fi
}
\makeatother

\IfFileExists{upquote.sty}{\usepackage{upquote}}{}
\definecolor{fgcolor}{rgb}{0.2, 0.2, 0.2}
\newcommand{\hlnumber}[1]{\textcolor[rgb]{0,0,0}{#1}}%
\newcommand{\hlfunctioncall}[1]{\textcolor[rgb]{0.501960784313725,0,0.329411764705882}{\textbf{#1}}}%
\newcommand{\hlstring}[1]{\textcolor[rgb]{0.6,0.6,1}{#1}}%
\newcommand{\hlkeyword}[1]{\textcolor[rgb]{0,0,0}{\textbf{#1}}}%
\newcommand{\hlargument}[1]{\textcolor[rgb]{0.690196078431373,0.250980392156863,0.0196078431372549}{#1}}%
\newcommand{\hlcomment}[1]{\textcolor[rgb]{0.180392156862745,0.6,0.341176470588235}{#1}}%
\newcommand{\hlroxygencomment}[1]{\textcolor[rgb]{0.43921568627451,0.47843137254902,0.701960784313725}{#1}}%
\newcommand{\hlformalargs}[1]{\textcolor[rgb]{0.690196078431373,0.250980392156863,0.0196078431372549}{#1}}%
\newcommand{\hleqformalargs}[1]{\textcolor[rgb]{0.690196078431373,0.250980392156863,0.0196078431372549}{#1}}%
\newcommand{\hlassignement}[1]{\textcolor[rgb]{0,0,0}{\textbf{#1}}}%
\newcommand{\hlpackage}[1]{\textcolor[rgb]{0.588235294117647,0.709803921568627,0.145098039215686}{#1}}%
\newcommand{\hlslot}[1]{\textit{#1}}%
\newcommand{\hlsymbol}[1]{\textcolor[rgb]{0,0,0}{#1}}%
\newcommand{\hlprompt}[1]{\textcolor[rgb]{0.2,0.2,0.2}{#1}}%

\usepackage{framed}
\makeatletter
\newenvironment{kframe}{%
 \def\at@end@of@kframe{}%
 \ifinner\ifhmode%
  \def\at@end@of@kframe{\end{minipage}}%
  \begin{minipage}{\columnwidth}%
 \fi\fi%
 \def\FrameCommand##1{\hskip\@totalleftmargin \hskip-\fboxsep
 \colorbox{shadecolor}{##1}\hskip-\fboxsep
     % There is no \\@totalrightmargin, so:
     \hskip-\linewidth \hskip-\@totalleftmargin \hskip\columnwidth}%
 \MakeFramed {\advance\hsize-\width
   \@totalleftmargin\z@ \linewidth\hsize
   \@setminipage}}%
 {\par\unskip\endMakeFramed%
 \at@end@of@kframe}
\makeatother

\definecolor{shadecolor}{rgb}{.97, .97, .97}
\definecolor{messagecolor}{rgb}{0, 0, 0}
\definecolor{warningcolor}{rgb}{1, 0, 1}
\definecolor{errorcolor}{rgb}{1, 0, 0}
\newenvironment{knitrout}{}{} % an empty environment to be redefined in TeX

\usepackage{alltt}
\usepackage{url}
\usepackage[sc]{mathpazo}
\usepackage{geometry}
\geometry{verbose,tmargin=2.5cm,bmargin=2.5cm,lmargin=2.5cm,rmargin=2.5cm}
\setcounter{secnumdepth}{2}
\setcounter{tocdepth}{2}
\usepackage[unicode=true,pdfusetitle,
 bookmarks=true,bookmarksnumbered=true,bookmarksopen=true,bookmarksopenlevel=2,
 breaklinks=false,pdfborder={0 0 1},backref=false,colorlinks=false]{hyperref}
\hypersetup{
 pdfstartview={XYZ null null 1}}
\usepackage{breakurl}

\begin{document}

\title{\bf A package for transparent and reproducible statistical
  yearbook}
\author{Michael C. J. Kao\\ Food and Agricultural Organization of the
  United Nation}
\date{}

\maketitle
\tableofcontents





\begin{knitrout}
\definecolor{shadecolor}{rgb}{0.969, 0.969, 0.969}\color{fgcolor}\begin{kframe}


{\ttfamily\noindent\bfseries\color{errorcolor}{\#\# Error: there is no package called 'Rgraphviz'}}\end{kframe}
\end{knitrout}


\section{Introduction}
The idea of using R and \LaTeX was initiated by Adam Prakash and
Mattieu Stigler of the FAO of the United Nation, this has now been
extended and become the standard procedure for producing the
statistical year book for the both the global and the regional
statistical year book of the FAO.

The automation along with the R package brings tremendous amount of
benefits, speeding up the production process and open up the data and
methodology to the general public.

In this small document we will illustrate the production process and
demonstrate how the use of the package can increase transparency and
sustainability.

\begin{knitrout}
\definecolor{shadecolor}{rgb}{0.969, 0.969, 0.969}\color{fgcolor}\begin{kframe}


{\ttfamily\noindent\bfseries\color{errorcolor}{\#\# Error: "graphNEL" is not a defined class}}

{\ttfamily\noindent\bfseries\color{errorcolor}{\#\# Error: could not find function "addEdge"}}

{\ttfamily\noindent\bfseries\color{errorcolor}{\#\# Error: could not find function "addEdge"}}

{\ttfamily\noindent\bfseries\color{errorcolor}{\#\# Error: could not find function "addEdge"}}

{\ttfamily\noindent\bfseries\color{errorcolor}{\#\# Error: could not find function "addEdge"}}

{\ttfamily\noindent\bfseries\color{errorcolor}{\#\# Error: could not find function "addEdge"}}

{\ttfamily\noindent\bfseries\color{errorcolor}{\#\# Error: could not find function "addEdge"}}

{\ttfamily\noindent\bfseries\color{errorcolor}{\#\# Error: could not find function "addEdge"}}

{\ttfamily\noindent\bfseries\color{errorcolor}{\#\# Error: could not find function "addEdge"}}

{\ttfamily\noindent\bfseries\color{errorcolor}{\#\# Error: could not find function "addEdge"}}

{\ttfamily\noindent\bfseries\color{errorcolor}{\#\# Error: could not find function "addEdge"}}

{\ttfamily\noindent\bfseries\color{errorcolor}{\#\# Error: could not find function "addEdge"}}

{\ttfamily\noindent\bfseries\color{errorcolor}{\#\# Error: could not find function "addEdge"}}

{\ttfamily\noindent\bfseries\color{errorcolor}{\#\# Error: could not find function "addEdge"}}

{\ttfamily\noindent\bfseries\color{errorcolor}{\#\# Error: could not find function "addEdge"}}

{\ttfamily\noindent\bfseries\color{errorcolor}{\#\# Error: could not find function "addEdge"}}

{\ttfamily\noindent\bfseries\color{errorcolor}{\#\# Error: could not find function "addEdge"}}

{\ttfamily\noindent\bfseries\color{errorcolor}{\#\# Error: could not find function "addEdge"}}

{\ttfamily\noindent\bfseries\color{errorcolor}{\#\# Error: could not find function "addEdge"}}

{\ttfamily\noindent\bfseries\color{errorcolor}{\#\# Error: could not find function "addEdge"}}

{\ttfamily\noindent\bfseries\color{errorcolor}{\#\# Error: could not find function "addEdge"}}

{\ttfamily\noindent\bfseries\color{errorcolor}{\#\# Error: could not find function "addEdge"}}

{\ttfamily\noindent\bfseries\color{errorcolor}{\#\# Error: could not find function "addEdge"}}

{\ttfamily\noindent\bfseries\color{errorcolor}{\#\# Error: could not find function "addEdge"}}

{\ttfamily\noindent\bfseries\color{errorcolor}{\#\# Error: could not find function "addEdge"}}

{\ttfamily\noindent\bfseries\color{errorcolor}{\#\# Error: could not find function "addEdge"}}

{\ttfamily\noindent\bfseries\color{errorcolor}{\#\# Error: could not find function "addEdge"}}

{\ttfamily\noindent\bfseries\color{errorcolor}{\#\# Error: could not find function "addEdge"}}

{\ttfamily\noindent\bfseries\color{errorcolor}{\#\# Error: could not find function "addEdge"}}

{\ttfamily\noindent\bfseries\color{errorcolor}{\#\# Error: could not find function "addEdge"}}

{\ttfamily\noindent\bfseries\color{errorcolor}{\#\# Error: could not find function "addEdge"}}

{\ttfamily\noindent\bfseries\color{errorcolor}{\#\# Error: could not find function "addEdge"}}

{\ttfamily\noindent\bfseries\color{errorcolor}{\#\# Error: could not find function "addEdge"}}

{\ttfamily\noindent\bfseries\color{errorcolor}{\#\# Error: could not find function "addEdge"}}

{\ttfamily\noindent\bfseries\color{errorcolor}{\#\# Error: could not find function "addEdge"}}

{\ttfamily\noindent\bfseries\color{errorcolor}{\#\# Error: could not find function "addEdge"}}

{\ttfamily\noindent\bfseries\color{errorcolor}{\#\# Error: could not find function "addEdge"}}

{\ttfamily\noindent\bfseries\color{errorcolor}{\#\# Error: could not find function "addEdge"}}

{\ttfamily\noindent\bfseries\color{errorcolor}{\#\# Error: object 'gd4' not found}}

{\ttfamily\noindent\bfseries\color{errorcolor}{\#\# Error: object 'gd4' not found}}

{\ttfamily\noindent\bfseries\color{errorcolor}{\#\# Error: object 'gd4' not found}}

{\ttfamily\noindent\bfseries\color{errorcolor}{\#\# Error: could not find function "renderGraph"}}\end{kframe}
\end{knitrout}


In the work flow diagram, square represents input files, eclipse
represnts R functions and diamonds are output files. The process
describes how the input files, intermediate outputs and R functions are
used to generate the final statistical year book of the FAO.

\subsection{Installations}

\section{Detail}
\subsection{Input files (In square)}
The Meta Data file and the construction file are temporary as they will
eventually be replaced and maintained in data bases to avoid manual fill
in. Nevertheless, the diseemination will always be manual as it is
specific to each book.

\subsubsection{Meta data file}
File containing information about all indicators, the source and download
path. Every indicator used in the statistical year book should have an
entry regardless whether it is extracted or computed. These information
has to be manually filled in currently, but in the near future we hope
to have a database to store these information.


\begin{table}[!ht]
  \centering
  \begin{tabular}{l|p{8cm}}
    \hline
    Column name & Description\\
    \hline
    DATA\_KEY & The FAO name of the variable\\
    SERIES\_NAME & The name and description of the variable\\
    SERIES\_NAME\_SHORT & Sames as SERIES\_NAME, just shorter and is
    mainly used for labels\\
    QUANTITY & The multiplier factor for the unit (e.g. 1000)\\
    ORIG\_UNIT & The unit the variable is measured in\\
    INFO & Additional information about the variable\\
    OWNER & The owner or provider of the data.\\
    SOURCE & The original source of the data\\
    DATA\_TYPE & Whether the data is download as raw or constructed
    based on the construction file.\\
    SQL & All the SQL related fields are used to specify the download
    path for FAO data.\\
    WDINAME & The indicator name for the World Bank Development Index.\\
    COMMENTS & Additional information.\\
    \hline
  \end{tabular}
\end{table}


\subsubsection{Construction file}
This file contains two important element, the aggregation rule for how
variables should be aggregated for both territorial or regional. Second,
the construction rule of indicators to be computed in house.

\begin{table}[!ht]
  \centering
  \begin{tabular}{l|p{8cm}}
    \hline
    Column name & Description\\
    \hline
    DATA\_KEY\_PROC & The name of the constructed variable, if missing
    then default names are assigned based on the type of construction\\
    DATA\_KEY\_ORIG1 & The first variable used for construction\\
    DATA\_KEY\_ORIG2 & Same as above, for shares this is the total variable\\
    CONSTRUCTION\_TYPE & The type of construction (e.g. share, growth)\\
    GROWTH\_RATE\_FREQ & The frequency of the growth rate (1 for annual,
    10 for 10 year growth rate vice versa)\\
    AGGREGATION & Not sure if this should be here anymore\\
    COMMENTS & Additional Information\\
    \hline
  \end{tabular}
\end{table}


%% \clearpage
%% \subsubsection{Dissemination file}
%% The dissemination file contains information of how to produce charts,
%% maps, tables and at the same time provide the labels, color and location
%% of the object within the book. It is also used to produce the \LaTeX
%% file for the report.

%% \begin{table}[!ht]
%%   \begin{tabular}{l|p{8cm}}
%%     \hline
%%     Column name & Description\\
%%     \hline
%%     PART\_CHAPTER\_NUMBER & The name of the plot\\
%%     PART & The location of the plot in the book\\
%%     TOPIC\_NAME & \\
%%     TOPIC\_NAME\_ABB & \\
%%     SPREAD\_NUMBERorTABLE\_NUMBER & \\
%%     TeX\_ENVIRONMENT & \\
%%     COLUMN\_NUMBERorOBJECT\_ORDER & \\
%%     MANUAL & \\
%%     OBJECT & Type of dissemination, chart, map or table\\
%%     TYPE & Obsolete\\
%%     SUBTYPE & Obsolete\\
%%     AREA & The regional classfication to be used\\
%%     TABLE\_NAMEorDESCRIP\_OBJECTorBULLET & Tile for the disseminated object\\
%%     COLUMN\_NAMEorCAPTION & The caption of the disseminated object\\
%%     TABLE\_SEP1 & \\
%%     TABLE\_SEP2 & \\
%%     YEAR & \\
%%     YearStart & The starting year of the plot or table\\
%%     YearEnd & The end date for the object\\
%%     Interval &\\
%%     QUANTITY & The quantity unit\\
%%     ORIG\_UNIT & The unit of the plot\\
%%     NUMBER\_OF\_VARIABLES & The number of variables used in the plot/chart.\\
%%     AGG & \\
%%     DATA\_KEY ($1\sim10$) & The name of the variables to be plotted.\\
%%     comment & \\
%%     \hline
%%   \end{tabular}
%% \end{table}

\subsection{FAOcountryProfile and FAOregionProfile}
These two in-built data sets comes along with the FAOSYB package and
contains information about country and regions.

\section{Step by step demonstration}

\subsection{Read the input files}
Lets read in the meta data file first

\begin{knitrout}
\definecolor{shadecolor}{rgb}{0.969, 0.969, 0.969}\color{fgcolor}\begin{kframe}
\begin{alltt}
meta.lst = \hlfunctioncall{read.FAOSYBmeta}(file = \hlstring{"Metadata.csv"})
\end{alltt}


{\ttfamily\noindent\bfseries\color{errorcolor}{\#\# Error: could not find function "read.FAOSYBmeta"}}\begin{alltt}
\hlfunctioncall{names}(meta.lst)
\end{alltt}


{\ttfamily\noindent\bfseries\color{errorcolor}{\#\# Error: object 'meta.lst' not found}}\end{kframe}
\end{knitrout}



From the structure, we can see that the object has five lists
contained. The first object contains the full file, while the WDI,
FAOSTAT list are the subsets of the full file which will be used to
enable automized download of World Bank and FAOSTAT data. While OTHER
contain a list of variables that have to be manually downloaded and
processed. The final element is the unit of the data where it will be
used to scale the data to the raw unit without any multiplier factor
(e.g. 1,000 Kilogram to just Kilogram)

Now let us read in the construction file which contains the rule for
constructin new variables.

\begin{knitrout}
\definecolor{shadecolor}{rgb}{0.969, 0.969, 0.969}\color{fgcolor}\begin{kframe}
\begin{alltt}
con.df = \hlfunctioncall{read.FAOSYBcon}(file = \hlstring{"Construction.csv"})
\end{alltt}


{\ttfamily\noindent\bfseries\color{errorcolor}{\#\# Error: could not find function "read.FAOSYBcon"}}\end{kframe}
\end{knitrout}




\subsection{Download data from FAOSTAT}
First lets download data from the FAOSTAT API
\url{http://faostat.fao.org/}, the SQL codes contained in ``FAOSTAT''
list of the meta data are required to download data from FAOSTAT. A name
has to be given to each of the variable.

\begin{knitrout}
\definecolor{shadecolor}{rgb}{0.969, 0.969, 0.969}\color{fgcolor}\begin{kframe}
\begin{alltt}
FAO.df = \hlfunctioncall{with}(meta.lst[[\hlstring{"FAOSTAT"}]],
    \hlfunctioncall{getFAOtoSYB}(name = DATA_KEY,
                domainCode = SQL_DOMAIN_CODE, elementCode = SQL_ELEMENT_CODE,
                itemCode = SQL_ITEM_CODE))
\end{alltt}
\end{kframe}
\end{knitrout}


The output contains three lists with the territory/country level data
stored in the first, aggregates in the second and the error messages in
the second. The user can examine the result to find out which variables
have been downloaded successfully, and which have failed.

\begin{knitrout}
\definecolor{shadecolor}{rgb}{0.969, 0.969, 0.969}\color{fgcolor}\begin{kframe}
\begin{alltt}
\hlfunctioncall{names}(FAO.df)
\end{alltt}
\begin{verbatim}
## [1] "FAOST_CODE"   "ISO2_WB_CODE" "Year"         "arableLand"   "cerealExp"   
## [6] "cerealProd"
\end{verbatim}
\end{kframe}
\end{knitrout}



%% This is just to check whether the data from FAO are complete and it
%% seems at least there is not data for Monaco (140)




%% Looks like there is no over-lapping between 351 and 357, but there
%% is not Taiwan.




\subsection{Download data from the World Bank API}
Now lets use the information in the ``WDI'' list to download data from
the World Bank Indicator API. More information about the API can be
found on the official website \url{http://data.worldbank.org/}

To use the \emph{getWDItoSYB} function for extracting data from the
World Bank API, only the indicator name is required (e.g. SP.POP.TOTL
for population). Several options are available for customising the
output, first the name argument allows one to assign different name to
the indicator if one wish to use a more meaningful name or a different
naming scheme. Second, the meta data of the variables can be downloaded
at the same time, and has the option to be printed and saved. Finally,
the World Bank provides aggregates such as ``World'', ``Heavily indebt
poor country (HIPC)'', if these are not required then the dropAggregate
argument will only preserve the lowest level data.

\begin{knitrout}
\definecolor{shadecolor}{rgb}{0.969, 0.969, 0.969}\color{fgcolor}\begin{kframe}
\begin{alltt}
WB.df = \hlfunctioncall{with}(meta.lst[[\hlstring{"WDI"}]], \hlfunctioncall{getWDItoSYB}(indicator = WDINAME,
    name = DATA_KEY, getMetaData = FALSE, saveMetaData = FALSE))
\end{alltt}
\end{kframe}
\end{knitrout}


The structure of the output is similar to the object downloaded by
\emph{getFAOtoSYB} except that the country coding system is in
``ISO2\_WB\_CODE'' scheme rather than FAO coding scheme.

\begin{knitrout}
\definecolor{shadecolor}{rgb}{0.969, 0.969, 0.969}\color{fgcolor}\begin{kframe}
\begin{alltt}
\hlfunctioncall{names}(WB.df)
\end{alltt}
\begin{verbatim}
## [1] "ISO2_WB_CODE"    "UN_CODE"         "Country"         "Year"           
## [5] "totalPopulation" "GDPUSD"
\end{verbatim}
\end{kframe}
\end{knitrout}


Nevertheless, the package also come along with the
\emph{translateCountryCode} function which enables the user to translate
between different coding system contained in the data
\textbf{FAOcountryProfile}. Otherwise, the same mechanism is also
provided by the \emph{mergeSYB} function.

\begin{knitrout}
\definecolor{shadecolor}{rgb}{0.969, 0.969, 0.969}\color{fgcolor}\begin{kframe}
\begin{alltt}
WBcountry.df = \hlfunctioncall{translateCountryCode}(WB.df$data, from = \hlstring{"ISO2_WB_CODE"},
  to = \hlstring{"FAOST_CODE"})
\end{alltt}


{\ttfamily\noindent\bfseries\color{errorcolor}{\#\# Error: 'by' must specify uniquely valid column(s)}}\end{kframe}
\end{knitrout}


\subsection{Working with manual data}
It is inevitable to deal with manual data from different sources,
below we give a brief demonstration of how to use some helper function
within the package to ease the job.

\begin{knitrout}
\definecolor{shadecolor}{rgb}{0.969, 0.969, 0.969}\color{fgcolor}\begin{kframe}
\begin{alltt}
wrpc.df = \hlfunctioncall{read.csv}(file = \hlstring{"WaterResourcesPerCapita.csv"}, header = TRUE,
  na.strings = \hlstring{"[[:space:]]"}, stringsAsFactors = FALSE)

\hlcomment{## Convert string spaces to empty and coerce to numeric.}
wrpc.df[, \hlstring{"AQ.WAT.WATPCP.MC.NO"}] = \hlfunctioncall{as.numeric}(\hlfunctioncall{gsub}(\hlstring{"[[:space:]]"}, \hlstring{""},
         wrpc.df[, \hlstring{"AQ.WAT.WATPCP.MC.NO"}]))
\end{alltt}
\end{kframe}
\end{knitrout}


Since the data from AQUASTAT and some other sources does not provide a
coding system, the \emph{fillFAOcode} helps to fill in the FAOSTAT code
based on the names matched in the FAOcountryProfile. However, certain
countries may not be matched and requires manual input.

\begin{knitrout}
\definecolor{shadecolor}{rgb}{0.969, 0.969, 0.969}\color{fgcolor}\begin{kframe}
\begin{alltt}
\hlcomment{## Use the fillFAOCOde function to partially fill FAOST_CODE}
wrpcFill.df <- \hlfunctioncall{fillFAOCode}(wrpc.df, countryNameColumn = \hlstring{"Country"})
\end{alltt}


{\ttfamily\noindent\bfseries\color{errorcolor}{\#\# Error: could not find function "fillFAOCode"}}\begin{alltt}

\hlcomment{## Extract the country that was not matched by the fillFAOCode function.}
missFAOcode <- \hlfunctioncall{unique}(wrpcFill.df[\hlfunctioncall{is.na}(wrpcFill.df$FAOST_CODE), \hlstring{"Country"}])
\end{alltt}


{\ttfamily\noindent\bfseries\color{errorcolor}{\#\# Error: object 'wrpcFill.df' not found}}\begin{alltt}

\hlcomment{## Find the code manually, some are not matched because they are multiply}
\hlcomment{## matched (e.g. China) and requires identification of which country they}
\hlcomment{## actually refer to.}
manualFAOcode <- \hlfunctioncall{c}(12, 351, 45, 47, 107, 167, 250, 56, 75, 120, 127, 150, 158, 171, 
    185, 206, 212, 225, 215, 231, 249)
manual.df <- \hlfunctioncall{data.frame}(Country = missFAOcode, NEW_FAOST_CODE = manualFAOcode)
\end{alltt}


{\ttfamily\noindent\bfseries\color{errorcolor}{\#\# Error: object 'missFAOcode' not found}}\begin{alltt}
tmp <- \hlfunctioncall{merge}(wrpcFill.df, manual.df, by = \hlstring{"Country"}, all.x = TRUE)
\end{alltt}


{\ttfamily\noindent\bfseries\color{errorcolor}{\#\# Error: object 'wrpcFill.df' not found}}\begin{alltt}
tmp[\hlfunctioncall{is.na}(tmp$FAOST_CODE), \hlstring{"FAOST_CODE"}] <- tmp[\hlfunctioncall{is.na}(tmp$FAOST_CODE), \hlstring{"NEW_FAOST_CODE"}]
\end{alltt}


{\ttfamily\noindent\bfseries\color{errorcolor}{\#\# Error: object 'tmp' not found}}\begin{alltt}
wrpcFinal.df <- tmp[, \hlfunctioncall{c}(\hlstring{"FAOST_CODE"}, \hlstring{"Year"}, \hlstring{"AQ.WAT.WATPCP.MC.NO"})]
\end{alltt}


{\ttfamily\noindent\bfseries\color{errorcolor}{\#\# Error: object 'tmp' not found}}\end{kframe}
\end{knitrout}



\subsection{Merge data from different sources}
The merge function provided in the package has in built translation, so
as long as the column name of the coding scheme matches the name in the
\textbf{FAOcountryProfile}, the function will recognise the coding
system and merge them together. There is one thing to note, there will
be two entries for China since the definition of China is different
between FAO and World Bank.

\begin{knitrout}
\definecolor{shadecolor}{rgb}{0.969, 0.969, 0.969}\color{fgcolor}\begin{kframe}
\begin{alltt}
allData.df = \hlfunctioncall{mergeSYB}(FAO.df$data,
  WB.df$data[, -\hlfunctioncall{grep}(\hlstring{"Country"}, \hlfunctioncall{colnames}(WB.df$data))], wrpcFinal.df)
\end{alltt}


{\ttfamily\noindent\bfseries\color{errorcolor}{\#\# Error: Unknown country code in x, check FAOcountryProfile for supported country code}}\end{kframe}
\end{knitrout}


After merging the data, we would like to scale the data to its base
unit. This is because the unit can differ during the dissemination step
and thus working with the base unit without any multiplier simplifies
the process. For example, to present the table we want to show the value
in Kilogram. However, we would want to show 100 kilogram when they are
aggregated in to region. When the variable is not at the base unit we
would have to find the conversion multipler, but when all the variable
are scale to base unit then we can simply just multipler and get what
every unit we like.

\subsection{Scale data to base unit}
It is recommended to scale the data to base units (e.g. use kg instead
of 1,000 kg). Using the base unit avoids confusion later in the
dissemintation step where a plot may require different scales.

\begin{knitrout}
\definecolor{shadecolor}{rgb}{0.969, 0.969, 0.969}\color{fgcolor}\begin{kframe}


{\ttfamily\noindent\bfseries\color{errorcolor}{\#\# Error: could not find function "translateUnit"}}\begin{alltt}
scaleData.df = \hlfunctioncall{toBaseUnit}(allData.df, meta.lst[[\hlstring{"UNIT_MULTI"}]])
\end{alltt}


{\ttfamily\noindent\bfseries\color{errorcolor}{\#\# Error: could not find function "toBaseUnit"}}\end{kframe}
\end{knitrout}






%% The following list of countries are reported but not in the M49
%% definition.




\subsection{Construct new variables using existing variables}
Using the rules provided by the construction file to construct new
variables such as growth and shares.

\begin{knitrout}
\definecolor{shadecolor}{rgb}{0.969, 0.969, 0.969}\color{fgcolor}\begin{kframe}
\begin{alltt}
newVar.df = \hlfunctioncall{with}(con.df[con.df$CONSTRUCTION_TYPE %in%
    \hlfunctioncall{c}(\hlstring{"share"}, \hlstring{"growth"}, \hlstring{"change"}, \hlstring{"index"}), ],
    \hlfunctioncall{constructSYB}(scaleData.df, origVar1 = DATA_KEY_CONSTR1,
        origVar2 = DATA_KEY_CONSTR2, newVarName = DATA_KEY_PROC,
        constructType = CONSTRUCTION_TYPE, grFreq = GROWTH_RATE_FREQ))
\end{alltt}


{\ttfamily\noindent\bfseries\color{errorcolor}{\#\# Error: object 'con.df' not found}}\end{kframe}
\end{knitrout}


\begin{knitrout}
\definecolor{shadecolor}{rgb}{0.969, 0.969, 0.969}\color{fgcolor}\begin{kframe}
\begin{alltt}
\hlfunctioncall{names}(newVar.df)
\end{alltt}


{\ttfamily\noindent\bfseries\color{errorcolor}{\#\# Error: object 'newVar.df' not found}}\end{kframe}
\end{knitrout}


Again, the output gives two list with the first containing the processed
data and the newly constructed variables in the same data frame. While
the second list contain the result of the construction.


\subsection{Aggregate and subset data into M49 definition}
The \emph{aggCountry} function allow one to aggregate territory entries
into country or higher level classification based on the relatinoship
specified.

The following example shows the relationship between the FAO definition
and the M49 definition. It is important to specify the correct method
and weighting variable otherwise the aggregate will produce erroneous
output. The \emph{keepUnspecified} option will retain the entries that
are not included in the M49 definition which will be useful to analyse
what has been omitted.



%% Use 41 for China in M49
\begin{knitrout}
\definecolor{shadecolor}{rgb}{0.969, 0.969, 0.969}\color{fgcolor}\begin{kframe}
\begin{alltt}
\hlcomment{## The water resources per capita does not have an aggregation method}
\hlcomment{## yet so we will use the mean.}
con.df[\hlfunctioncall{grep}(\hlstring{"AQ.WAT.WAT"}, con.df$DATA_KEY_PROC), \hlstring{"AGGREGATION"}] = \hlstring{"mean"}
\end{alltt}


{\ttfamily\noindent\bfseries\color{errorcolor}{\#\# Error: object 'con.df' not found}}\begin{alltt}

crdf = \hlfunctioncall{data.frame}(FAOST_CODE = FAOcountryProfile[, \hlstring{"FAOST_CODE"}, ],
                  UN_CODE = FAOcountryProfile[, \hlstring{"MOTHER_M49_CODE"}, ])

country.df = \hlfunctioncall{with}(con.df[!\hlfunctioncall{is.na}(con.df$AGGREGATION), ],
    \hlfunctioncall{aggCountry}(newVar.df$data,
               relationDF = crdf,
               aggVar = DATA_KEY_PROC, aggMethod = AGGREGATION,
               weightVar = DATA_KEY_WEIGHT, keepUnspecified = FALSE))
\end{alltt}


{\ttfamily\noindent\bfseries\color{errorcolor}{\#\# Error: object 'con.df' not found}}\begin{alltt}

country.df = \hlfunctioncall{merge}(FAOcountryProfile[,
    \hlfunctioncall{c}(\hlstring{"UN_CODE"}, \hlstring{"OFFICIAL_FAO_NAME"})], country.df)
\end{alltt}


{\ttfamily\noindent\bfseries\color{errorcolor}{\#\# Error: object 'country.df' not found}}\begin{alltt}


\hlcomment{## Countries that have been aggregated into unspecified}
FAOcountryProfile[FAOcountryProfile$FAOST_CODE %in%
                  \hlfunctioncall{c}(15, 51, 62, 151, 164, 186, 205, 206, 228, 243, 248),
                  \hlstring{"OFFICIAL_FAO_NAME"}]
\end{alltt}
\end{kframe}
\end{knitrout}


After obtaining the data, the package also contain functions for
exploring data. We strongly advise users to explore the territory
level data before proceeding with any further manipulation.

\subsection{Explore sparsity}
Below are two example of how to use the \emph{sparsityHeatMap} to see
how the missingness occur in each variable by country and time. The plot
is grouped into four panels, the first three panel groups the country by
their similarity of value while the last panel shows country which have
not reported any value.

The population data is very complete and missing only at the later years
for some country.


\begin{knitrout}
\definecolor{shadecolor}{rgb}{0.969, 0.969, 0.969}\color{fgcolor}\begin{kframe}
\begin{alltt}
explore.df= \hlfunctioncall{merge}(FAOcountryProfile[, \hlfunctioncall{c}(\hlstring{"UN_CODE"}, \hlstring{"LAB_NAME"})],
  country.df, by = \hlstring{"UN_CODE"})
\end{alltt}


{\ttfamily\noindent\bfseries\color{errorcolor}{\#\# Error: object 'country.df' not found}}\begin{alltt}
explore.df = \hlfunctioncall{subset}(explore.df, Year %in% 1990:2012)
\end{alltt}


{\ttfamily\noindent\bfseries\color{errorcolor}{\#\# Error: object 'explore.df' not found}}\begin{alltt}
\hlfunctioncall{sparsityHeatMap}(explore.df, country = \hlstring{"LAB_NAME"}, year = \hlstring{"Year"},
                var = \hlstring{"SP.POP.TOTL"})
\end{alltt}


{\ttfamily\noindent\bfseries\color{errorcolor}{\#\# Error: could not find function "sparsityHeatMap"}}\end{kframe}
\end{knitrout}


\clearpage
On the other hand, we can see the agricultural value add per work is
very sparse. In addition, a large amount of countries does not even have
any data.
\begin{knitrout}
\definecolor{shadecolor}{rgb}{0.969, 0.969, 0.969}\color{fgcolor}\begin{kframe}
\begin{alltt}
\hlfunctioncall{sparsityHeatMap}(explore.df, country = \hlstring{"LAB_NAME"}, year = \hlstring{"Year"},
                var = \hlstring{"NV.AGR.TOTL.ZS"})
\end{alltt}


{\ttfamily\noindent\bfseries\color{errorcolor}{\#\# Error: could not find function "sparsityHeatMap"}}\end{kframe}
\end{knitrout}


\clearpage
\begin{knitrout}
\definecolor{shadecolor}{rgb}{0.969, 0.969, 0.969}\color{fgcolor}\begin{kframe}
\begin{alltt}
\hlfunctioncall{sparsityHeatMap}(explore.df, country = \hlstring{"LAB_NAME"}, year = \hlstring{"Year"},
                var = \hlstring{"AQ.WAT.WATPCP.MC.NO"})
\end{alltt}


{\ttfamily\noindent\bfseries\color{errorcolor}{\#\# Error: could not find function "sparsityHeatMap"}}\end{kframe}
\end{knitrout}


\clearpage
The panel data plot may requires a large screen or scaled and saved if
the number of countries are large. The advantage of this plot allows one
to identify the behaviour of this variable, in particular if one was to
build models or carrying out imputation. The chracteristics that govern
the variable and the transferability of country information determines
what type of model is available.

\begin{knitrout}
\definecolor{shadecolor}{rgb}{0.969, 0.969, 0.969}\color{fgcolor}\begin{kframe}
\begin{alltt}
\hlfunctioncall{tsPanel}(explore.df, country = \hlstring{"LAB_NAME"}, year = \hlstring{"Year"},
        var = \hlstring{"SP.POP.TOTL"}, ylab = \hlstring{"Total Population"})
\end{alltt}


{\ttfamily\noindent\bfseries\color{errorcolor}{\#\# Error: object 'explore.df' not found}}\end{kframe}
\end{knitrout}



\begin{knitrout}
\definecolor{shadecolor}{rgb}{0.969, 0.969, 0.969}\color{fgcolor}\begin{kframe}
\begin{alltt}
\hlfunctioncall{tsPanel}(explore.df, country = \hlstring{"LAB_NAME"}, year = \hlstring{"Year"},
        var = \hlstring{"AQ.WAT.WATPCP.MC.NO"}, ylab = \hlstring{"Water resources per capita"})
\end{alltt}


{\ttfamily\noindent\bfseries\color{errorcolor}{\#\# Error: object 'explore.df' not found}}\end{kframe}
\end{knitrout}



%% Do a jitter bar graph

\subsection{Imputation for regional aggregates}
We will use linear interpolation between two successful observations
and last observation carry forward for extrapolation.
\begin{knitrout}
\definecolor{shadecolor}{rgb}{0.969, 0.969, 0.969}\color{fgcolor}\begin{kframe}
\begin{alltt}
\hlfunctioncall{source}(\hlstring{"~/Dropbox/SYBproject/FAOSYB Package/Package Test/temporaryScript.R"})
\end{alltt}


{\ttfamily\noindent\bfseries\color{errorcolor}{\#\# Error: cannot open the connection}}\begin{alltt}
imputed.df = \hlfunctioncall{sybImpute}(var = \hlfunctioncall{colnames}(newVar.df$data[, -\hlfunctioncall{c}(1:2)]),
    country = \hlstring{"FAOST_CODE"}, year = \hlstring{"Year"}, data = newVar.df$data)
\end{alltt}


{\ttfamily\noindent\bfseries\color{errorcolor}{\#\# Error: could not find function "sybImpute"}}\begin{alltt}

\hlcomment{## We will use data up to 2012}
imputed.df = \hlfunctioncall{subset}(imputed.df, Year <= 2012)
\end{alltt}


{\ttfamily\noindent\bfseries\color{errorcolor}{\#\# Error: object 'imputed.df' not found}}\end{kframe}
\end{knitrout}


\subsection{Aggregate into regions}
After exploring the territory level data, we can aggregate the data to
obtain the region and the global figure.

The aggregation on countries and region has been seperated rather than
sequential, the main reason for this is due to the time dimension of
region which is not present in country level data. For example, we
would not aggregate the old data of Sudan (206) into either Sudan
(276) and South sudan (277). However, when we compute the regional
aggregates all these three countries will be aggregated into Africa
and then the world.

\begin{knitrout}
\definecolor{shadecolor}{rgb}{0.969, 0.969, 0.969}\color{fgcolor}\begin{kframe}
\begin{alltt}

rrdf = FAOcountryProfile[, \hlfunctioncall{c}(\hlstring{"FAOST_CODE"}, \hlstring{"UNSD_MACRO_REG_CODE"})]
\end{alltt}


{\ttfamily\noindent\bfseries\color{errorcolor}{\#\# Error: undefined columns selected}}\begin{alltt}

\hlcomment{## The Pacific region are all NA, because the threshhold is not}
\hlcomment{## met. This is why we need the insignificant country threshold rule.}
region.df = \hlfunctioncall{with}(con.df[!\hlfunctioncall{is.na}(con.df$AGGREGATION), ],
    \hlfunctioncall{aggRegion}(Data = imputed.df,
              relationDF = rrdf,
              aggVar = DATA_KEY_PROC, aggMethod = AGGREGATION,
              weightVar = DATA_KEY_WEIGHT, keepUnspecified = FALSE))
\end{alltt}


{\ttfamily\noindent\bfseries\color{errorcolor}{\#\# Error: object 'con.df' not found}}\begin{alltt}

\hlcomment{## Check why zero the MACRO code is fill incorrectly.}
\hlcomment{## FAOcountryProfile[FAOcountryProfile$FAOST_CODE %in%}
\hlcomment{##                   c(15, 51, 62, 151, 164, 186, 206, 228, 248),}
\hlcomment{##                   "OFFICIAL_FAO_NAME"]}


region.df = \hlfunctioncall{merge}(\hlfunctioncall{na.omit}(\hlfunctioncall{unique}(FAOcountryProfile[,
    \hlfunctioncall{c}(\hlstring{"UNSD_MACRO_REG_CODE"}, \hlstring{"UNSD_MACRO_REG"})])), region.df, all.y = TRUE)
\end{alltt}


{\ttfamily\noindent\bfseries\color{errorcolor}{\#\# Error: undefined columns selected}}\begin{alltt}
\hlfunctioncall{colnames}(region.df)[1:2] = \hlfunctioncall{c}(\hlstring{"UN_CODE"}, \hlstring{"OFFICIAL_FAO_NAME"})


wrdf = \hlfunctioncall{data.frame}(UN_CODE = \hlfunctioncall{c}(2, 9, 19, 142, 150),
                  World = \hlfunctioncall{rep}(0, 5))

\hlcomment{## Check how to remove the zeros, and also fix two UNSD_MACRO_REG_CODE}
world.df = \hlfunctioncall{with}(con.df[!\hlfunctioncall{is.na}(con.df$AGGREGATION), ],
    \hlfunctioncall{aggRegion}(Data = region.df,
              relationDF = wrdf,
              aggVar = DATA_KEY_PROC, aggMethod = AGGREGATION,
              weightVar = DATA_KEY_WEIGHT, keepUnspecified = FALSE))
\end{alltt}


{\ttfamily\noindent\bfseries\color{errorcolor}{\#\# Error: object 'con.df' not found}}\begin{alltt}

\hlfunctioncall{colnames}(world.df)[1] = \hlstring{"UN_CODE"}
\end{alltt}


{\ttfamily\noindent\bfseries\color{errorcolor}{\#\# Error: object 'world.df' not found}}\begin{alltt}

world.df = \hlfunctioncall{data.frame}(UN_CODE = world.df[, 1], OFFICIAL_FAO_NAME = \hlstring{"World"},
                      world.df[, 2:\hlfunctioncall{NCOL}(world.df)])
\end{alltt}


{\ttfamily\noindent\bfseries\color{errorcolor}{\#\# Error: object 'world.df' not found}}\begin{alltt}

\hlcomment{## Combine everything}
all.df = \hlfunctioncall{areaBind}(territory = country.df, region = region.df,
    world = world.df)
\end{alltt}


{\ttfamily\noindent\bfseries\color{errorcolor}{\#\# Error: could not find function "areaBind"}}\end{kframe}
\end{knitrout}







%% Check why the variable numbers are different in computing aggregates
%% and the number of variables in the data

\end{document}
